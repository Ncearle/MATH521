\documentclass[10pt,letterpaper]{scrartcl}
\usepackage{amsfonts,amsmath,amssymb}
\usepackage{dsfont,fontawesome}
\usepackage{braket,mathtools,siunitx}
\usepackage[hidelinks]{hyperref}
\usepackage{textcomp,url}
\usepackage{xcolor,graphicx,tikz,pgfplots}
\definecolor{shadecolor}{rgb}{0.9,0.9,0.9}
\usetikzlibrary{arrows}
\pgfplotsset{compat=1.12}
\usepackage{listings,enumerate}
\usepackage{booktabs,tabularx,longtable,multicol}
\usepackage[inner=2cm,outer=2cm,top=2cm,bottom=2.3cm]{geometry}

\pagestyle{empty}
\setlength{\parindent}{0pt}
\setlength{\parskip}{6pt}

\newcommand{\dx}{\;\mathrm{d}x}
\newcommand{\ds}{\;\mathrm{d}s}
\renewcommand{\div}{\operatorname{div}}

\begin{document}

\begin{minipage}{.2\textwidth}
\includegraphics[width=42pt]{ubc-logo.png}
\end{minipage}
\hfill
\begin{minipage}{.75\textwidth}
\setlength{\parskip}{6pt}
\begin{flushright}
{
\sffamily
\textbf{MATH521: Numerical Analysis of Partial Differential Equations}\\
Winter 2018/19, Term 2

Due Date: Thursday, 7 February 2019\\
Timm Treskatis
}
\end{flushright}
\end{minipage}

\section*{Homework Assignment 5}

Please submit the following files as indicated below: \hfill \faFileCodeO \: source code \hfill \faFilePdfO \: PDF file \hfill \faFilePictureO \: image file \hfill \faFileMovieO \: video file

\paragraph*{Question 1 $\vert$ 2 marks $\vert$ \faFilePdfO}

In this assignment, we consider the linear elasticity problem
\begin{equation}\label{eq:le}
\begin{aligned}
-c\Delta u + a u &= f &&\text{in } \Omega\\
u &= g && \text{on } \partial\Omega
\end{aligned}
\end{equation}
on a polygonal domain $\Omega$. The function $u$ can be interpreted as the elongation of a rubber membrane over the $x_1x_2$-plane. The boundary values $g$ prescribe the elongation on $\partial\Omega$, e.g. by means of a wire frame construction in which the membrane has been fixed. The real number $c>0$ is the stiffness of the rubber material, $a>0$ is a constant proportional to its mass density and the inhomogeneity $f$ models external forces that act on the membrane.

\begin{enumerate}[(a)]
\item Show that under the assumption of homogeneous boundary conditions, $g=0$, the discretisation of \eqref{eq:le} with linear finite elements reads
\begin{equation*}
(cK^h + aM^h) \vec{u}^h = \vec{f}^h
\end{equation*}
where
\begin{align*}
k_{ij}^h &= \int\limits_\Omega \nabla \phi_i^h \cdot \nabla \phi_j^h \dx\\
m_{ij}^h &= \int\limits_\Omega \phi_i^h \phi_j^h \dx\\
f_{i}^h &= \langle f, \phi_i^h\rangle_{H^{-1}(\Omega),H^1_0(\Omega)}\\
\phi_i^h &= \text{hat function centred at the $i$-th vertex}
\end{align*}
for $i,j = 1,\dots ,N$. $N$ is the number of effective degrees of freedom, i.e. the number of interior grid points which are not located on the boundary $\partial\Omega$.

Note that since the domain is assumed to be a polygon, we can cover it exactly with a triangulation $\mathcal{T}^h$ such that $\Omega = \Omega^h$ (there is no mismatch on the boundary).

\newpage

\mbox{}

\newpage
\item Show that the matrix $cK^h + aM^h$ of this linear system is symmetric positive definite. Is $M^h$ an $M$-matrix?
\end{enumerate}
\newpage

\paragraph*{Question 2 $\vert$ 3 marks $\vert$ \faFileCodeO}

Download the file \texttt{discretiseLinearElasticity.m}. We will turn this function into a finite-element solver for Problem \eqref{eq:le} next week. Today we implement some core components.

The files \texttt{video.mat} and \texttt{kiwi.mat} contain arrays \texttt{P}, \texttt{E} and \texttt{T} which define a triangulation on a polygonal computational domain $\Omega^h$. Note that some versions of \textsf{MATLAB}'s plotting functions from the PDE Toolbox require extra rows in \texttt{E} and \texttt{T}. If you are not using the PDE Toolbox, then you may delete all but the first two rows of \texttt{E} and all but the first three rows of \texttt{T}, as described in the video on triangulations.

To import the variables from \texttt{video.mat} or \texttt{kiwi.mat} into a structure \texttt{msh}, you may use the \texttt{load} command. In \textsf{Python}, use \texttt{scipy.io.loadmat} and subtract 1 from all entries in \texttt{E} and \texttt{T} for \textsf{Python}'s zero-based indexing.

\begin{enumerate}[(a)]
\item Recall that the $k$-th triangle in the triangulation has the vertices $\mathtt{T(1,k)}$, $\mathtt{T(2,k)}$ and $\mathtt{T(3,k)}$ and you may look up their coordinates in the matrix \texttt{P}. For example, $\mathtt{P(:,T(1,k))}$ returns the two coordinates of the first vertex in the $k$-th triangle.

Complete the main function and the \texttt{assembleMass} subfunction. (Optional: Can you do it without \texttt{for} loops?)

\emph{Hint:} In \textsf{GNU Octave / MATLAB}, the command \texttt{sparse} may be helpful. The corresponding command in \textsf{Python} is \texttt{scipy.sparse.csr\_matrix}. Even though this is not documented, if multiple values have the same row and column indices, the \texttt{csr\_matrix} command automatically sums up these values. In \textsf{GNU Octave} and \textsf{MATLAB}, this behaviour of the \texttt{sparse} command is documented.

\item \faFilePictureO \: Write a script \texttt{hw5.m} to plot the triangular mesh and the sparsity pattern of the mass matrix that the function \texttt{discretiseLinearElasticity} returns (you don't have to remove the rows/columns corresponding to boundary points). Do this for both data sets \texttt{video.mat} and \texttt{kiwi.mat}. Make sure your plots are not distorted by using the \texttt{axis equal} command.


\emph{Hint:} In installations of \textsf{MATLAB} with the PDE Toolbox, the command \texttt{pdemesh} may be helpful. In \textsf{GNU Octave} and \textsf{MATLAB} without the PDE Toolbox, the command \texttt{trimesh} may be helpful. In \textsf{Python}, use \texttt{plot\_trisurf} with zero $z$-values. Note that \texttt{trimesh} and \texttt{plot\_trisurf} need the transpose of \texttt{T}.
\item \faFilePictureO \: Add extra commands to this script to plot your favourite function $u^h$ on the kiwi domain and compute its $L^2$-norm. Constant functions are not allowed! Make sure your plots are not distorted.

\emph{Hint:} The commands \texttt{pdeplot}, \texttt{trisurf} or \texttt{plot\_trisurf}, respectively, may be helpful.
\end{enumerate}

\paragraph*{Your Learning Progress $\vert$ 0 marks, but -1 mark if unanswered $\vert$ \faFilePdfO}
What is the one most important thing that you have learnt from this assignment?

% These lines are just here for the folks who submit handwritten answers. As you seem to type up your answers, just delete the lines :)
\vspace*{3mm}
\hrulefill

\vspace*{3mm}
\hrulefill

\vspace*{3mm}
\hrulefill

What is the most substantial new insight that you have gained from this course this week? Any \emph{aha moment}?

\vspace*{3mm}
\hrulefill

\vspace*{3mm}
\hrulefill

\vspace*{3mm}
\hrulefill

\end{document}