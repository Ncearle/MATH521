\documentclass[10pt,letterpaper]{scrartcl}
\usepackage{amsfonts,amsmath,amssymb}
\usepackage{dsfont,fontawesome}
\usepackage{braket,mathtools,siunitx}
\usepackage[hidelinks]{hyperref}
\usepackage{textcomp,url}
\usepackage{xcolor,graphicx,tikz,pgfplots}
\definecolor{shadecolor}{rgb}{0.9,0.9,0.9}
\usetikzlibrary{arrows}
\pgfplotsset{compat=1.12}
\usepackage{listings,enumerate}
\usepackage{booktabs,tabularx,longtable,multicol}
\usepackage[inner=2cm,outer=2cm,top=2cm,bottom=2.3cm]{geometry}

\pagestyle{empty}
\setlength{\parindent}{0pt}
\setlength{\parskip}{6pt}

\newcommand{\dx}{\;\mathrm{d}x}
\newcommand{\ds}{\;\mathrm{d}s}
\renewcommand{\div}{\operatorname{div}}

\begin{document}

\begin{minipage}{.2\textwidth}
\includegraphics[width=42pt]{ubc-logo.png}
\end{minipage}
\hfill
\begin{minipage}{.75\textwidth}
\setlength{\parskip}{6pt}
\begin{flushright}
{
\sffamily
\textbf{MATH521: Numerical Analysis of Partial Differential Equations}\\
Winter 2018/19, Term 2

Due Date: Thursday, 21 March 2019\\
Timm Treskatis
}
\end{flushright}
\end{minipage}

\section*{Homework Assignment 10}

Please submit the following files as indicated below: \hfill \faFileCodeO \: source code \hfill \faFilePdfO \: PDF file \hfill \faFilePictureO \: image file \hfill \faFileMovieO \: video file

\paragraph*{Question 1 $\vert$ 2 marks $\vert$ \faFilePdfO}

On the assignment page you can find videos of four animated solutions of the parabolic problem
\begin{equation}\tag{H}\label{eq:heat}
\begin{aligned}
\frac{\partial u}{\partial t}(t) - a \Delta u(t) &= f(t) && \text{in } Q = ]0,T[ \times \Omega\\
u(0) &= u_0 && \text{in } \Omega\\
\frac{\partial u}{\partial n} &= 0 && \text{on } \Sigma = ]0,T[ \times \partial \Omega
\end{aligned}
\end{equation}
with the data from Assignment 9. However, the initial condition has been replaced with the function
\begin{equation*}
u_0(x) =
\begin{dcases*}
50 & if $\lvert x - (1,1)^\top \rvert < 0.5$\\
20 & elsewhere
\end{dcases*}
\end{equation*}

Explain your observations, using the proper terminology.

\begin{itemize}
\item Crank-Nicolson method, $h \approx 1/50$, $\Delta t = 1/50$:

\vfill

\item Crank-Nicolson method, $h \approx 1/250$, $\Delta t = 1/50$:

\vfill

\item $\theta$-method with $\theta = 0.51$, $h \approx 1/250$, $\Delta t = 1/50$:

\vfill

\item TR-BDF2 method with $\alpha = 2-\sqrt{2}$, $h \approx 1/250$, $\Delta t = 1/50$:

\vfill
\end{itemize}
\newpage

\paragraph*{Question 2 $\vert$ 1 mark $\vert$ \faFilePdfO}

We have seen that the homogeneous wave equation
\begin{equation}\tag{W}\label{eq:wave}
\begin{aligned}
\frac{\partial^2 u}{\partial t^2} - c^2 \Delta u &= 0 && \text{in } Q = \left] 0,T\right[ \times \Omega\\
u(0) &= u_0 && \text{in } \Omega\\
\frac{\partial u}{\partial t}(0) &= v_0 && \text{in } \Omega\\
u &= 0 && \text{on } \Sigma = ]0,T[ \times \partial \Omega
\end{aligned}
\end{equation}
with propagation speed $c>0$ can equivalently be re-written as
\begin{equation}\tag{W'}\label{eq:waveSystem}
\begin{aligned}
\frac{\partial u}{\partial t} - v &= 0 && \text{in } Q = \left] 0,T\right[ \times \Omega\\
\frac{\partial v}{\partial t} - c^2 \Delta u &= 0 && \text{in } Q = \left] 0,T\right[ \times \Omega\\
u(0) &= u_0 && \text{in } \Omega\\
v(0) &= v_0 && \text{in } \Omega\\
u &= 0 && \text{on } \Sigma = ]0,T[ \times \partial \Omega\\
v &= 0 && \text{on } \Sigma = ]0,T[ \times \partial \Omega.
\end{aligned}
\end{equation}
Discretising with the $\theta$-method in time and linear finite elements in space leads to the coupled system for the vectors of nodal values $\vec{u}^h_+$ and $\vec{v}^h_+$
\begin{align*}
M^h \vec{u}^h_+ - \theta \Delta t M^h \vec{v}^h_+ &= M^h \vec{u}^h_\circ + (1-\theta) \Delta t M^h\vec{v}^h_\circ\\
\theta \Delta t c^2 K^h \vec{u}^h_+ + M^h \vec{v}^h_+ &= -(1-\theta) \Delta t c^2 K^h \vec{u}^h_\circ + M^h \vec{v}^h_\circ
\end{align*}
which has to be solved at every time step. Show that this is equivalent to the two smaller, successively solvable problems
\begin{align*}
\left(M^h + \left(\theta \Delta t c\right)^2 K^h\right) \vec{u}^h_+ &= M^h \left( \vec{u}^h_\circ + \Delta t \vec{v}^h_\circ\right) - \left( \theta\left(1-\theta\right) \left(\Delta t c\right)^2 \right)K^h \vec{u}^h_\circ\\
M^h \vec{v}^h_+ &= M^h \vec{v}^h_\circ - \Delta t c^2 K^h \left( \theta \vec{u}^h_+ + \left(1-\theta\right) \vec{u}^h_\circ\right).
\end{align*}

\newpage

\paragraph*{Question 3 $\vert$ 2 marks}

\begin{enumerate}[(a)]
\item \faFileCodeO \: Download and complete the \textsf{FEniCS} script \texttt{hw10.py} to solve Problem \eqref{eq:wave} with the data provided.
\item \faFilePdfO \:  Solve the wave equation
\begin{itemize}
\item with the Crank-Nicolson method
\item with the backward Euler method
\item with the forward Euler method
\end{itemize}
and look at the solutions in \textsf{ParaView}.

\emph{Hint:} Use the `Warp by Scalar' filter, re-scale the colour map to the range $[-1,1]$ and tick the box `enable opacity mapping for surfaces' in the colour map editor.

Explain your observations. As always, take care to use the terminology correctly.

\vfill 
\end{enumerate}

\paragraph*{Your Learning Progress $\vert$ 0 marks, but -1 mark if unanswered $\vert$ \faFilePdfO}
What is the one most important thing that you have learnt from this assignment?

% These lines are just here for the folks who submit handwritten answers. As you seem to type up your answers, just delete the lines :)
\vspace*{3mm}
\hrulefill

\vspace*{3mm}
\hrulefill

\vspace*{3mm}
\hrulefill

What is the most substantial new insight that you have gained from this course this week? Any \emph{aha moment}?

\vspace*{3mm}
\hrulefill

\vspace*{3mm}
\hrulefill

\vspace*{3mm}
\hrulefill

\end{document}