\documentclass[10pt,letterpaper]{scrartcl}
\usepackage{amsfonts,amsmath,amssymb}
\usepackage{dsfont,fontawesome}
\usepackage{braket,mathtools,siunitx}
\usepackage[hidelinks]{hyperref}
\usepackage{textcomp,url}
\usepackage{xcolor,graphicx,tikz,pgfplots}
\definecolor{shadecolor}{rgb}{0.9,0.9,0.9}
\usetikzlibrary{arrows}
\pgfplotsset{compat=1.12}
\usepackage{listings,enumerate}
\usepackage{booktabs,tabularx,longtable,multicol}
\usepackage[inner=2cm,outer=2cm,top=2cm,bottom=2.3cm]{geometry}

\pagestyle{empty}
\setlength{\parindent}{0pt}
\setlength{\parskip}{6pt}

\newcommand{\dx}{\;\mathrm{d}x}
\newcommand{\ds}{\;\mathrm{d}s}
\renewcommand{\div}{\operatorname{div}}

\begin{document}

\begin{minipage}{.2\textwidth}
\includegraphics[width=42pt]{ubc-logo.png}
\end{minipage}
\hfill
\begin{minipage}{.75\textwidth}
\setlength{\parskip}{6pt}
\begin{flushright}
{
\sffamily
\textbf{MATH521: Numerical Analysis of Partial Differential Equations}\\
Winter 2018/19, Term 2

Due Date: Thursday, 14 March 2019\\
Timm Treskatis
}
\end{flushright}
\end{minipage}

\section*{Homework Assignment 9}

Please submit the following files as indicated below: \hfill \faFileCodeO \: source code \hfill \faFilePdfO \: PDF file \hfill \faFilePictureO \: image file \hfill \faFileMovieO \: video file

\paragraph*{Question 1 $\vert$ 2 marks $\vert$ \faFilePdfO}

We consider the initial boundary value problem for the heat equation
\begin{equation}\tag{H}\label{eq:heat}
\begin{aligned}
\frac{\partial u}{\partial t}(t) - a \Delta u(t) &= f(t) && \text{in } Q = ]0,T[ \times \Omega\\
u(0) &= u_0 && \text{in } \Omega\\
\frac{\partial u}{\partial n} &= 0 && \text{on } \Sigma = ]0,T[ \times \partial \Omega
\end{aligned}
\end{equation}
where $u$ is a temperature field, $u_0$ an initial temperature distribution, the diffusion-like parameter $a > 0$ the heat conductivity of the material, $f$ a source term e.g. due to thermal radiation and $T>0$ a final time. The homogeneous Neumann boundary conditions mean that the domain $\Omega$ is perfectly insulated so that no thermal energy is radiated into the environment.

The $\theta$-method is a class of Runge-Kutta schemes for integrating ODEs of the form
\begin{equation*}
\dot{U} = F(t,U)
\end{equation*}
by using the iteration
\begin{equation*}
U_+ = U_\circ + \Delta t \left( \theta F(t_+,U_+) + (1-\theta) F(t_\circ,U_\circ) \right).
\end{equation*}
The parameter $\theta \in [0,1]$ can be interpreted as the `degree of implicitness', since $\theta = 0$ gives the forward Euler method, $\theta = \frac{1}{2}$ the Crank-Nicolson method (aka implicit trapezium rule in the ODE context) and $\theta = 1$ the backward Euler method.

For the discretisation in space, we apply linear finite elements.

Show that if the spatial triangulation $\mathcal{T}^h$ remains fixed, then both the method of lines and Rothe's method lead to the same discrete problems
\begin{equation*}
\left( M^h + \theta \Delta t a K^h\right) \vec{u}^h_+ = \left( M^h - (1-\theta)\Delta t a K^h\right) \vec{u}^h_\circ + \Delta t \left( \theta \vec{f}^h_+ + (1-\theta) \vec{f}^h_\circ \right).
\end{equation*}

You don't have to include any details about the components of the discrete vectors and matrices. We all know what they are!

\newpage

\mbox{}

\vspace{11cm}

\paragraph*{Question 2 $\vert$ 3 marks}

\begin{enumerate}[(a)]
\item \faFileMovieO \: The \textsf{FEniCS} script \texttt{hw9.py} implements the backward Euler method for Problem \eqref{eq:heat}. Starting from room temperature ($u_0 \equiv 20$), the bottom left corner of a metal piece $\Omega$ with conductivity parameter $a=0.1$ is held over a flame for one second, then the flame is extinguished. This is modelled by
\begin{equation*}
f(t,x) =
\begin{dcases}
200e^{-5x_1^2 - 2x_2^2} & t \leq 1\\
0 & t > 1
\end{dcases}
\end{equation*}
Complete the missing commands to compute the evolution of the temperature field over the first five seconds using a time step size of $\Delta t = 10^{-2}$.

Save your results as a video, using a frame rate such that the video time is equal to the physical time. You don't have to submit any other files for this part of Question 2.

\emph{Hint:} Open the PVD-file in \textsf{ParaView}, click the \emph{Apply} button and make sure that in the \emph{View} menu the \emph{Color Map Editor} is highlighted. Then select a reasonable colour map
\begin{center}
\includegraphics[scale=1]{colourmap.png}
\end{center}
and re-scale the colour values
\begin{center}
\includegraphics[scale=1]{rescale.png}
\end{center}
to the range $[20,160]$. Use the same range for the following questions, too.
\item \faFileCodeO \: Generalise this script to implement the $\theta$-method from Question 1. Check whether setting $\theta = 1$ still gives you the same results. Using the same parameters as in Question 2(a), solve the problem with the Crank-Nicolson method and the forward Euler method. What do you observe?

\vfill

\item \faFilePdfO \: Solve the problem with the forward Euler method again up to time $T=0.1$, once with $\Delta t = 1.25 \times 10^{-4}$ and once with $\Delta t = 10^{-4}$. Explain your observations, using the relevant terminology.

\vfill

\end{enumerate}

\paragraph*{Your Learning Progress $\vert$ 0 marks, but -1 mark if unanswered $\vert$ \faFilePdfO}
What is the one most important thing that you have learnt from this assignment?

% These lines are just here for the folks who submit handwritten answers. As you seem to type up your answers, just delete the lines :)
\vspace*{3mm}
\hrulefill

\vspace*{3mm}
\hrulefill

\vspace*{3mm}
\hrulefill

What is the most substantial new insight that you have gained from this course this week? Any \emph{aha moment}?

\vspace*{3mm}
\hrulefill

\vspace*{3mm}
\hrulefill

\vspace*{3mm}
\hrulefill

\end{document}