\documentclass[10pt,letterpaper]{scrartcl}
\usepackage{amsfonts,amsmath,amssymb}
\usepackage{dsfont,fontawesome}
\usepackage{braket,mathtools,siunitx}
\usepackage[hidelinks]{hyperref}
\usepackage{textcomp,url}
\usepackage{xcolor,graphicx,tikz,pgfplots}
\definecolor{shadecolor}{rgb}{0.9,0.9,0.9}
\usetikzlibrary{arrows}
\pgfplotsset{compat=1.12}
\usepackage{listings,enumerate}
\usepackage{booktabs,tabularx,longtable,multicol}
\usepackage[inner=2cm,outer=2cm,top=2cm,bottom=2.3cm]{geometry}

\pagestyle{empty}
\setlength{\parindent}{0pt}
\setlength{\parskip}{6pt}

\newcommand{\dx}{\;\mathrm{d}x}
\newcommand{\ds}{\;\mathrm{d}s}
\renewcommand{\div}{\operatorname{div}}

\begin{document}

\begin{minipage}{.2\textwidth}
\includegraphics[width=42pt]{ubc-logo.png}
\end{minipage}
\hfill
\begin{minipage}{.75\textwidth}
\setlength{\parskip}{6pt}
\begin{flushright}
{
\sffamily
\textbf{MATH521: Numerical Analysis of Partial Differential Equations}\\
Winter 2018/19, Term 2

Due Date: Thursday, 28 February 2019\\
Timm Treskatis
}
\end{flushright}
\end{minipage}

\section*{Homework Assignment 7}

Please submit the following files as indicated below: \hfill \faFileCodeO \: source code \hfill \faFilePdfO \: PDF file \hfill \faFilePictureO \: image file \hfill \faFileMovieO \: video file

\paragraph*{5 marks $\vert$ \faFilePdfO}

Let $D>0$, $a\in\mathds{R}^2$, $r\geq 0$ and $f\in L^2(\Omega)$, where $\Omega\subset \mathds{R}^2$ is a convex, polygonal domain.

We use conforming linear finite elements with exact integration to solve the steady diffusion-advection-reaction problem
\begin{align*}
-D \Delta u + \div(au) + ru &= f && \text{in } \Omega\\
u &= 0 && \text{on } \partial\Omega.
\end{align*}
(Recall that the assumption of homogeneous boundary conditions is no loss of generality, since any inhomogeneous boundary conditions could be subtracted from $u$ to obtain the same PDE with homogeneous boundary conditions but a new source term.)

Follow the methodology from pp~67--68 in our notes to show that the numerical solution $u^h$ converges to $u$ at a linear rate in the $H^1$-norm and at a quadratic rate in the $L^2$-norm, provided that $u \in H^2(\Omega)$:
\begin{align}
\lVert u^h - u \rVert_{H^1(\Omega)} &\leq c h \lVert \nabla^2 u \rVert_{L^2(\Omega)}\\
\lVert u^h - u \rVert_{L^2(\Omega)} &\leq c h^2 \lVert \nabla^2 u \rVert_{L^2(\Omega)}.
\end{align}

\emph{Hints:}
\begin{enumerate}
\item To show that the nonsymmetric bilinear form of this elliptic operator is coercive in the $H^1$-norm, prove and then use that
\begin{equation}
\int\limits_\Omega (a\cdot \nabla u)v \dx = - \int\limits_\Omega (a\cdot \nabla v)u \dx \quad \text{for all } u,v \in H^1_0(\Omega).
\end{equation}
\item You may assume that
\begin{equation}
\lVert u \rVert_{H^2(\Omega)} \leq c \lVert f \rVert_{L^2(\Omega)}.
\end{equation}
\end{enumerate}

\newpage

\mbox{}

\newpage

\mbox{}

\newpage

\mbox{}

\vfill

\paragraph*{Your Learning Progress $\vert$ 0 marks, but -1 mark if unanswered $\vert$ \faFilePdfO}
What is the one most important thing that you have learnt from this assignment?

% These lines are just here for the folks who submit handwritten answers. As you seem to type up your answers, just delete the lines :)
\vspace*{3mm}
\hrulefill

\vspace*{3mm}
\hrulefill

\vspace*{3mm}
\hrulefill

What is the most substantial new insight that you have gained from this course this week? Any \emph{aha moment}?

\vspace*{3mm}
\hrulefill

\vspace*{3mm}
\hrulefill

\vspace*{3mm}
\hrulefill

\end{document}